\documentclass{article}
\usepackage[utf8]{inputenc}

\usepackage{fullpage}
\usepackage{titlesec}
\usepackage{amsthm}
\usepackage{amsmath}
\usepackage{amssymb}
\usepackage[portuguese]{babel}

\title{Resumo - Exame de Qualificação}
\author{IME USP\\Aluno: Rodrigo Aparecido Enju\\Orientador: Yoshiharu Kohayakawa }
\date{July 2019}

\usepackage{setspace}
\usepackage{datetime}
\usepackage{lineno}
\newcommand*\patchAmsMathEnvironmentForLineno[1]{%
\expandafter\let\csname old#1\expandafter\endcsname\csname #1\endcsname
\expandafter\let\csname oldend#1\expandafter\endcsname\csname end#1\endcsname
\renewenvironment{#1}%
{\linenomath\csname old#1\endcsname}%
{\csname oldend#1\endcsname\endlinenomath}}% 
\newcommand*\patchBothAmsMathEnvironmentsForLineno[1]{%
\patchAmsMathEnvironmentForLineno{#1}%
\patchAmsMathEnvironmentForLineno{#1*}}%
\AtBeginDocument{%
\patchBothAmsMathEnvironmentsForLineno{equation}%
\patchBothAmsMathEnvironmentsForLineno{align}%
\patchBothAmsMathEnvironmentsForLineno{flalign}%
\patchBothAmsMathEnvironmentsForLineno{alignat}%
\patchBothAmsMathEnvironmentsForLineno{gather}%
\patchBothAmsMathEnvironmentsForLineno{multline}%
}

\begin{document}
\linenumbers 
\shortdate
\yyyymmdddate
\settimeformat{ampmtime}
\date{\today, \currenttime}
\renewcommand{\abstractname}{Resumo}
\onehalfspace

\maketitle

Erd\H{o}s provou que para todo par de inteiros positivos $k$ e $g$, existe um grafo com número cromático pelo menos $k$ e cintura pelo menos $g$ \cite{erdos1959graph}. Erd\H{o}s e Hajnal conjecturam a existência de um inteiro $f(k,g)$ para todo par de inteiros positivos $k$ e $g$ tal que todo grafo com número cromático pelo menos $f(k,g)$ contém um subgrafo com número cromático pelo menos $k$ e cintura pelo menos $g$ \cite{erdos1969conj}.

O foco do projeto será o estudo da conjectura de Erd\H{o}s e Hajnal. Serão abordados alguns resultados parciais e famílias de grafos para as quais a conjectura é verdadeira.

Dentre os resultados a serem estudados temos a demonstração que a conjectura vale para $g = 4$ e $k$ arbitrário, ou seja, que para todo $k$ existe um inteiro~$f(k,4)$ tal que todo grafo com número cromático pelo menos $f(k,4)$ contém um subgrafo livre de triângulos e com número cromático pelo menos $k$. 

Assim como será demonstrado o caso $k=3$ e $g$ arbitrário, ou seja, para todo $g$ existe um $f(3,g)$ tal que todo grafo com número cromático pelo menos $f(3,g)$ contém um subgrafo com número cromático pelo menos~$3$ e cintura pelo menos $g$, em particular, tal subgrafo é um circuito ímpar de comprimento pelo menos~$g$.

Mohar e Wu demonstraram outro importante resultado, mostrando que a conjectura é verdadeira para a família de grafos de Kneser \cite{mohar2015triangle}.

\bibliographystyle{ieeetr}
\bibliography{ref}

\endgroup
\end{document}

%%% Local Variables:
%%% mode: latex
%%% eval: (auto-fill-mode t)
%%% eval: (LaTeX-math-mode t)
%%% eval: (flyspell-mode t)
%%% TeX-master: t
%%% End:
