\chapter{Considerações Finais}
\label{cap:consfinais}


%%
%% Parágrafo introdutório. %%
%%


Embora uma solução definitiva para a conjectura de Erd\H{o}s e Hajnal ainda não seja conhecida, resultados para diferentes famílias de grafos

%%
%% Resumo dos capítulos anteriores talvez. %%
%%

Segue um breve resumo dos capítulos anteriores. No Capitulo \ref{cap:consiniciais} vimos alguns exemplos de construções de grafos livres de triângulos e com número cromático alto, especificamente, a construção de Tutte \cite{descartes1947three}, a construção de Zykov \cite{zykov1949some}, a construção de Mycielski \cite{mycielski1955coloriage} e a construção por planos projetivos \cite{codenotti2000some}. Vimos também o teorema de Erd\H{o}s \cite{erdos1959graph}, resultado fundamental para o estudo do número cromático e cintura de grafos.

No Capítulo \ref{cap:livretriangulos} vimos dois casos ``base'' da conjectura de Erd\H{o}s e Hajnal, no sentido de que caso uma indução seja possível sobre os dois parâmetros, os casos com os menores valores de $k$ e $g$ foram provados. Em particular, se $g=4$ (ou seja, se queremos subgrafos livres de triângulos), a conjetura é verdadeira \cite{rodl1977chromatic}, e se $k=3$ (ou seja, se queremos subgrafos com circuitos ímpares), a conjectura é verdadeira.

No Capítulo \ref{cap:kneser} vimos que a família dos grafos de Kneser respeitam a conjectura de Erd\H{o}s e Hajnal \cite{mohar2015triangle}. Em particular mostramos por meio do Lema Local de Lovász que \textit{blow-ups} suficientemente grandes de grafos de Kneser contém subgrafos com número cromático e cintura grandes, e também mostramos que \textit{blow-ups} de grafos de Kneser são subgrafos de grafos de Kneser maiores.

No Capítulo \ref{cap:shift} vimos que a família dos \textit{shift graphs} de ordem $r$ respeitam a conjectura de Erd\H{o}s e Hajnal \cite{gabor2018cepa}. Consideramos também os \textit{type graphs}, apresentamos alguns breves resultados mostrando que para \textit{types} particulares, a conjectura de Erd\H{o}s e Hajnal é verdadeira para os \textit{type graphs}.

%%
%% Talvez comentar sobre porque acreditar que a conjectura é verdadeira. %%
%%

%%
%% Talvez comentar sobre possíveis continuações do tema. %%
%%

A conjectura de Erd\H{o}s e Hajnal permanece em aberto, e assim, é interessante o estudo de outras famílias de grafos. Uma possibilidade é procurar famílias de \textit{type graphs} que respeitam a conjectura de Erd\H{o}s e Hajnal com diferentes \textit{types}.

Outra possibilidade é considerar o número cromático fracionário, uma generalização do número cromático em que podemos colorir os vértices usando quantidades reais de cores. Mohar e Wu demonstraram em \cite{mohar2015triangle} a versão fracionária do teorema de R\"{o}dl.

No contexto da conjectura de Erd\H{o}s e Hajnal, uma questão natural é considerar a seguinte variação da conjectura

\begin{conjectura}
Para todo par de inteiros positivos $k$ e $g$, existe um inteiro $f(k,g)$ tal que todo grafo $G$ com $\chi_f(G) \geq f(k,g)$ contém um subgrafo $H$ com $\chi_f(H)\geq k$ e $g(H) \geq g$.
\end{conjectura}