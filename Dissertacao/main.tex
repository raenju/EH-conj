% Arquivo LaTeX de exemplo de dissertação/tese a ser apresentados à CPG do IME-USP
% 
% Versão 5: Sex Mar  9 18:05:40 BRT 2012
%
% Criação: Jesús P. Mena-Chalco
% Revisão: Fabio Kon e Paulo Feofiloff
%  
% Obs: Leia previamente o texto do arquivo README.txt

\documentclass[11pt,twoside,a4paper]{book}

% ---------------------------------------------------------------------------- %
% Pacotes 
\usepackage{float}
\usepackage{amsthm}
\usepackage{amsmath}
\usepackage{amssymb}
%\usepackage{fullpage}
%\usepackage{titlesec}
\usepackage[T1]{fontenc}
\usepackage[portuguese]{babel}
\usepackage[utf8]{inputenc}
\usepackage[pdftex]{graphicx}           % usamos arquivos pdf/png como figuras
\usepackage{setspace}                   % espaçamento flexível
\usepackage{indentfirst}                % indentação do primeiro parágrafo
\usepackage{makeidx}                    % índice remissivo
\usepackage[nottoc]{tocbibind}          % acrescentamos a bibliografia/indice/conteudo no Table of Contents
\usepackage{courier}                    % usa o Adobe Courier no lugar de Computer Modern Typewriter
\usepackage{type1cm}                    % fontes realmente escaláveis
\usepackage{listings}                   % para formatar código-fonte (ex. em Java)
\usepackage{titletoc}
%\usepackage[bf,small,compact]{titlesec} % cabeçalhos dos títulos: menores e compactos
\usepackage[fixlanguage]{babelbib}
\usepackage[font=small,format=plain,labelfont=bf,up,textfont=it,up]{caption}
\usepackage[usenames,svgnames,dvipsnames]{xcolor}
\usepackage[a4paper,top=2.54cm,bottom=2.0cm,left=2.0cm,right=2.54cm]{geometry} % margens
%\usepackage[pdftex,plainpages=false,pdfpagelabels,pagebackref,colorlinks=true,citecolor=black,linkcolor=black,urlcolor=black,filecolor=black,bookmarksopen=true]{hyperref} % links em preto
\usepackage[pdftex,unicode,plainpages=false,pdfpagelabels,pagebackref,colorlinks=true,citecolor=DarkGreen,linkcolor=NavyBlue,urlcolor=DarkRed,filecolor=green,bookmarksopen=true]{hyperref} % links coloridos
\usepackage[all]{hypcap}                % soluciona o problema com o hyperref e capitulos
%\usepackage[square,sort,nonamebreak,comma]{natbib}  % citação bibliográfica alpha (alpha-ime.bst)
\fontsize{60}{62}\usefont{OT1}{cmr}{m}{n}{\selectfont}
% ---------------------------------------------------------------------------- %
\usepackage{mathtools}
\usepackage{tikz}
\usetikzlibrary{decorations}
\usetikzlibrary{decorations.pathreplacing}
\usetikzlibrary{arrows}
\usetikzlibrary{arrows.meta, positioning, quotes}
\DeclarePairedDelimiter\ceil{\lceil}{\rceil}

\newif\ifveroriginal
\veroriginaltrue % true - ver para defesa, false - ver revisada
\newcommand{\datadefesa}{1/1/1999}

%\titleformat{\section}{\normalfont\Large}{\thesection}{1em}{}
\newtheorem{teorema}{Teorema}[section]
\newtheorem{lema}[teorema]{Lema}%[teorema]
\newtheorem{definicao}[teorema]{Definição}
\newtheorem{conjectura}{Conjectura}
\newtheorem{corolario}[teorema]{Corolário}
\newtheorem{afirmacao}[teorema]{Afirmação}
\newtheorem{fato}[teorema]{Fato}

\newcommand{\bbp}{\mathbb{P}}
\newcommand{\bbe}{\mathbb{E}}

\newcommand{\formattedtitle}{Título do trabalho a ser apresentado à \\
    CPG para a dissertação}
% ---------------------------------------------------------------------------- %
% Cabeçalhos similares ao TAOCP de Donald E. Knuth
\usepackage{fancyhdr}
\pagestyle{fancy}
\fancyhf{}
\renewcommand{\chaptermark}[1]{\markboth{\MakeUppercase{#1}}{}}
\renewcommand{\sectionmark}[1]{\markright{\MakeUppercase{#1}}{}}
\renewcommand{\headrulewidth}{0pt}

% ---------------------------------------------------------------------------- %
\graphicspath{{./figuras/}}             % caminho das figuras (recomendável)
\frenchspacing                          % arruma o espaço: id est (i.e.) e exempli gratia (e.g.) 
\urlstyle{same}                         % URL com o mesmo estilo do texto e não mono-spaced
\makeindex                              % para o índice remissivo
\raggedbottom                           % para não permitir espaços extra no texto
\fontsize{60}{62}\usefont{OT1}{cmr}{m}{n}{\selectfont}
\cleardoublepage
\normalsize

% ---------------------------------------------------------------------------- %
% Opções de listing usados para o código fonte
% Ref: http://en.wikibooks.org/wiki/LaTeX/Packages/Listings
\lstset{ %
language=Java,                  % choose the language of the code
basicstyle=\footnotesize,       % the size of the fonts that are used for the code
numbers=left,                   % where to put the line-numbers
numberstyle=\footnotesize,      % the size of the fonts that are used for the line-numbers
stepnumber=1,                   % the step between two line-numbers. If it's 1 each line will be numbered
numbersep=5pt,                  % how far the line-numbers are from the code
showspaces=false,               % show spaces adding particular underscores
showstringspaces=false,         % underline spaces within strings
showtabs=false,                 % show tabs within strings adding particular underscores
frame=single,	                % adds a frame around the code
framerule=0.6pt,
tabsize=2,	                    % sets default tabsize to 2 spaces
captionpos=b,                   % sets the caption-position to bottom
breaklines=true,                % sets automatic line breaking
breakatwhitespace=false,        % sets if automatic breaks should only happen at whitespace
escapeinside={\%*}{*)},         % if you want to add a comment within your code
backgroundcolor=\color[rgb]{1.0,1.0,1.0}, % choose the background color.
rulecolor=\color[rgb]{0.8,0.8,0.8},
extendedchars=true,
xleftmargin=10pt,
xrightmargin=10pt,
framexleftmargin=10pt,
framexrightmargin=10pt
}

% ---------------------------------------------------------------------------- %
% Corpo do texto
\begin{document}
\frontmatter 
% cabeçalho para as páginas das seções anteriores ao capítulo 1 (frontmatter)
\fancyhead[RO]{{\footnotesize\rightmark}\hspace{2em}\thepage}
\setcounter{tocdepth}{2}
\fancyhead[LE]{\thepage\hspace{2em}\footnotesize{\leftmark}}
\fancyhead[RE,LO]{}
\fancyhead[RO]{{\footnotesize\rightmark}\hspace{2em}\thepage}

\onehalfspacing  % espaçamento

% ---------------------------------------------------------------------------- %
% CAPA
% Nota: O título para as dissertações/teses do IME-USP devem caber em um 
% orifício de 10,7cm de largura x 6,0cm de altura que há na capa fornecida pela SPG.
\thispagestyle{empty}
\begin{center}
    \vspace*{2.3cm}
    \textbf{\Large{\formattedtitle}}\\
    
    \vspace*{1.2cm}
    \Large{Rodrigo Aparecido Enju}
    
    \vskip 2cm
    \textsc{
    Dissertação apresentada\\[-0.25cm] 
    ao\\[-0.25cm]
    Instituto de Matemática e Estatística\\[-0.25cm]
    da\\[-0.25cm]
    Universidade de São Paulo\\[-0.25cm]
    para\\[-0.25cm]
    obtenção do título\\[-0.25cm]
    de\\[-0.25cm]
    Mestre em Ciências}
    
    \vskip 1.5cm
    Programa: Ciência da Computação\\
    Orientador: Prof. Dr. Yoshiharu Kohayakawa

   	\vskip 1cm
    \normalsize{Durante o desenvolvimento deste trabalho o autor recebeu auxílio
    financeiro da CAPES}
    
    \vskip 0.5cm
    \normalsize{São Paulo, fevereiro de 2021}
\end{center}

% ---------------------------------------------------------------------------- %
% Página de rosto (SÓ PARA A VERSÃO DEPOSITADA - ANTES DA DEFESA)
% Resolução CoPGr 5890 (20/12/2010)
%
% IMPORTANTE:
%   Coloque um '%' em todas as linhas
%   desta página antes de compilar a versão
%   final, corrigida, do trabalho
%
%
\ifveroriginal
\newpage
\thispagestyle{empty}
    \begin{center}
        \vspace*{2.3 cm}
        \textbf{\Large{\formattedtitle}}\\
        \vspace*{2 cm}
    \end{center}

    \vskip 2cm

    \begin{flushright}
	Esta é a versão original da dissertação elaborada pelo\\
	candidato Rodrigo Aparecido Enju, tal como \\
	submetida à Comissão Julgadora.
    \end{flushright}

\pagebreak

% ---------------------------------------------------------------------------- %
% Página de rosto (SÓ PARA A VERSÃO CORRIGIDA - APÓS DEFESA)
% Resolução CoPGr 5890 (20/12/2010)
%
% Nota: O título para as dissertações/teses do IME-USP devem caber em um 
% orifício de 10,7cm de largura x 6,0cm de altura que há na capa fornecida pela SPG.
%
% IMPORTANTE:
%   Coloque um '%' em todas as linhas desta
%   página antes de compilar a versão do trabalho que será entregue
%   à Comissão Julgadora antes da defesa
%
%

\else
\newpage
\thispagestyle{empty}
    \begin{center}
        \vspace*{2.3 cm}
        \textbf{\Large{\formattedtitle}}\\
        \vspace*{2 cm}
    \end{center}

    \vskip 2cm

    \begin{flushright}
	Esta versão da dissertação contém as correções e alterações sugeridas\\
	pela Comissão Julgadora durante a defesa da versão original do trabalho,\\
	realizada em \datadefesa. Uma cópia da versão original está disponível no\\
	Instituto de Matemática e Estatística da Universidade de São Paulo.

    \vskip 2cm

    \end{flushright}
    \vskip 4.2cm

    \begin{quote}
    \noindent Comissão Julgadora:
    
    \begin{itemize}
		\item Profª. Drª. Nome Completo (orientadora) - IME-USP [sem ponto final]
		\item Prof. Dr. Nome Completo - IME-USP [sem ponto final]
		\item Prof. Dr. Nome Completo - IMPA [sem ponto final]
    \end{itemize}
      
    \end{quote}
\pagebreak
\fi


\pagenumbering{roman}     % começamos a numerar 

% ---------------------------------------------------------------------------- %
% Agradecimentos:
% Se o candidato não quer fazer agradecimentos, deve simplesmente eliminar esta página 
\chapter*{Agradecimentos}

% \minipage[t]{\dimexpr0.78\linewidth-2\fboxsep-2\fboxrule\relax}

Agradeço ao meu orientador, prof. Yoshiharu pela ajuda e paciência durante meu mestrado.

Agradeço à Larissa, ao Lucas e ao André pela companhia e pela ajuda ao longo de minha graduação e mestrado.

Agradeço à minha família pelo apoio.

Agradeço ao Gabriel e ao Tássio pela ajuda ao longo da pesquisa, em problemas que não sabia resolver, assim como por direções de como continuar.

Agradeço à Giulia, por me coorientar em minha conclusão da graduação e com isso indiretamente me encaminhar para a decisão de ingressar no mestrado.

Agradeço aos meus amigos da USP.

% \endminipage

% ---------------------------------------------------------------------------- %
% Resumo
\chapter*{Resumo}

\noindent ENJU, R. A. \textbf{Título do trabalho em português}. 
2021. 120 f.
Dissertação - Instituto de Matemática e Estatística,
Universidade de São Paulo, São Paulo, 2021.
\\

Um resultado de Erd\H{o}s demonstra a existência de grafos com número cromático e cintura arbitrariamente grandes. Temos então que um clique suficientemente grande contém um grafo com número cromático e cintura grandes como subgrafo, porém muitos grafos de interesse não necessariamente contém cliques grandes, então é interessante encontrar outra condição que garanta a existência de subgrafos com número cromático e cintura grandes. Uma conjectura de Erd\H{o}s e Hajnal diz que todo grafo com número cromático suficientemente grande contém um subgrafo com número cromático e cintura grandes. O objetivo deste trabalho é estudar tal conjectura.

O texto começa com uma breve apresentação de construções livres de triângulos. Em particular, é demonstrada uma construção de Codenotti, Pudlák e Resta, por meio de planos projetivos.

O tópico principal do texto começa com uma demonstração de R\"{o}dl de que todo grafo com número cromático suficientemente grande contém um subgrafo livre de triângulos e com número cromático grande. Em sequência, uma demonstração de que grafos com número cromático suficientemente grande contém algum circuito ímpar grande.

Apresentaremos também um resultado de Mohar e Wu, que demonstra que a família dos grafos de Kneser respeita a conjectura de Erd\H{o}s e Hajnal. Outro resultado apresentado é de Gábor Tardos, demonstrando que a família dos \textit{shift graphs} respeita a conjectura de Erd\H{o}s e Hajnal. E por fim apresentaremos alguns breves resultados sobre os \textit{type graphs}, mostrando casos que respeitam a conjectura de Erd\H{o}s e Hajnal.
\\
% Elemento obrigatório, constituído de uma sequência de frases concisas e
% objetivas, em forma de texto.  Deve apresentar os objetivos, métodos empregados,
% resultados e conclusões.  O resumo deve ser redigido em parágrafo único, conter
% no máximo 500 palavras e ser seguido dos termos representativos do conteúdo do
% trabalho (palavras-chave). 
% Texto texto texto texto texto texto texto texto texto texto texto texto texto
% texto texto texto texto texto texto texto texto texto texto texto texto texto
% texto texto texto texto texto texto texto texto texto texto texto texto texto
% texto texto texto texto texto texto texto texto texto texto texto texto texto
% texto texto texto texto texto texto texto texto texto texto texto texto texto
% texto texto texto texto texto texto texto texto.
% 

\noindent \textbf{Palavras-chave:} número cromático, cintura, grafo de Kneser, shift graph, type graph.

% ---------------------------------------------------------------------------- %
% Abstract
\chapter*{Abstract}
\noindent ENJU R. A. \textbf{Título do trabalho em inglês}. 
2021. 120 f.
Thesis - Instituto de Matemática e Estatística,
Universidade de São Paulo, São Paulo, 2021.
\\

A result by Erd\H{o}s shows that there exists graphs with arbitrarily large chromatic number and girth. Thus a sufficiently large clique contains a subgraph with large chromatic number and girth, but many graphs do not have a large clique, hence it is interesting to find a different condition that guarantees the existence of a subgraph with large chromatic number and girth. A conjecture by Erd\H{o}s and Hajnal states that every graph with sufficiently large chromatic number contains a subgraph with large chromatic number and girth. The objective of this text is to study this conjecture.


The text begins with a brief discussion of triangle-free constructions. In particular, we show a construction by Codenotti, Pudlák and Resta, based on projective planes.

The main topic begins with a proof by R\"{o}dl, that every graph with sufficiently large chromatic number contains a triangle-free subgraph with large chromatic number. We follow with a proof that every graph with sufficiently large chromatic number contains a large odd cycle.

We then show a result by Mohar and Wu, which shows that the Kneser graphs follow the Erd\H{o}s-Hajnal conjecture. Another result by Gábor Tardos proves that shift graphs also follow the Erd\H{o}s-Hajnal conjecture. Finally, we show some brief results about type graphs, showing some cases that follow the Erd\H{o}s-Hajnal conjecture.
\\


% Elemento obrigatório, elaborado com as mesmas características do resumo em
% língua portuguesa. De acordo com o Regimento da Pós- Graduação da USP (Artigo
% 99), deve ser redigido em inglês para fins de divulgação. 
% Text text text text text text text text text text text text text text text text
% text text text text text text text text text text text text text text text text
% text text text text text text text text text text text text text text text text
% text text text text text text text text text text text text.
% \\

\noindent \textbf{Keywords:} chromatic number, girth, Kneser graph, shift graph, type graph.

% ---------------------------------------------------------------------------- %
% Sumário
\tableofcontents    % imprime o sumário

% ---------------------------------------------------------------------------- %
% \chapter{Lista de Abreviaturas}
% \begin{tabular}{ll}
%          CFT         & Transformada contínua de Fourier (\emph{Continuous Fourier Transform})\\
%          DFT         & Transformada discreta de Fourier (\emph{Discrete Fourier Transform})\\
%         EIIP         & Potencial de interação elétron-íon (\emph{Electron-Ion Interaction Potentials})\\
%         STFT         & Tranformada de Fourier de tempo reduzido (\emph{Short-Time Fourier Transform})\\
% \end{tabular}

% ---------------------------------------------------------------------------- %
% \chapter{Lista de Símbolos}
% \begin{tabular}{ll}
%         $\omega$    & Frequência angular\\
%         $\psi$      & Função de análise \emph{wavelet}\\
%         $\Psi$      & Transformada de Fourier de $\psi$\\
% \end{tabular}

% ---------------------------------------------------------------------------- %
% Listas de figuras e tabelas criadas automaticamente
\listoffigures            
%\listoftables            

% ---------------------------------------------------------------------------- %
% Capítulos do trabalho
\mainmatter

% cabeçalho para as páginas de todos os capítulos
\fancyhead[RE,LO]{\thesection}
\setlength{\headheight}{16pt}

%\singlespacing              % espaçamento simples
\onehalfspacing            % espaçamento um e meio

\input introducao
\input consiniciais
\input livretriangulo
%\input circimpares
\input kneser
\input shift
\input consfinais
%\input cap-introducao        % associado ao arquivo: 'cap-introducao.tex'
%\input cap-conceitos         % associado ao arquivo: 'cap-conceitos.tex'
%\input cap-conclusoes        % associado ao arquivo: 'cap-conclusoes.tex'

% cabeçalho para os apêndices
\renewcommand{\chaptermark}[1]{\markboth{\MakeUppercase{\appendixname\ \thechapter}} {\MakeUppercase{#1}} }
\fancyhead[RE,LO]{}
\appendix

\include{ape-conjuntos}      % associado ao arquivo: 'ape-conjuntos.tex'

% ---------------------------------------------------------------------------- %
% Bibliografia
%\backmatter \singlespacing   % espaçamento simples
\bibliographystyle{amsplain}% citação bibliográfica alpha
\bibliography{ref}  % associado ao arquivo: 'bibliografia.bib'

% ---------------------------------------------------------------------------- %
% Índice remissivo
% \index{TBP|see{periodicidade região codificante}}
% \index{DSP|see{processamento digital de sinais}}
% \index{STFT|see{transformada de Fourier de tempo reduzido}}
% \index{DFT|see{transformada discreta de Fourier}}
% \index{Fourier!transformada|see{transformada de Fourier}}

% \printindex   % imprime o índice remissivo no documento 

\end{document}

%%% Local Variables:
%%% mode: latex
%%% eval: (auto-fill-mode t)
%%% eval: (LaTeX-math-mode t)
%%% eval: (flyspell-mode t)
%%% TeX-master: t
%%% End: