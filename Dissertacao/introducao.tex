\chapter{Introdução}
\label{cap:introducao}

Uma possível abordagem para o estudo de grafos é por meio de seu número cromático e de sua cintura, os quais são parâmetros relacionados à distribuição de suas arestas.

\begin{definicao}
O número cromático de um grafo $G$, denotado por $\chi(G)$, é o menor número de cores necessárias para colorir os vértices de $G$ de forma que vértices adjacentes tenham cores distintas.
\end{definicao}

\begin{definicao}
A cintura de um grafo $G$, denotada por $g(G)$, é o comprimento de um menor circuito de $G$.
\end{definicao}

Um exemplo fundamental de grafos com número cromático grande são os grafos completos, grafos em que todo par de vértices são adjacentes, e logo cada vértice precisa de uma cor única. De forma semelhante, uma forma simples de aumentar o número cromático de um grafo é inserir novas arestas, teremos que o número cromático se aproxima do número de vértices conforme o grafo se aproxima de um grafo completo.

Dessa forma, parece natural que para um grafo ter número cromático grande, ele contenha uma grande concentração de arestas.

Por outro lado, grafos com cintura grande têm baixa concentração de arestas, pois seus circuitos não podem conter cordas que gerem circuitos pequenos. Dessa forma temos que localmente, um grafo de cintura grande se assemelha a uma árvore.

Pode parecer intuitivo que um grafo com número cromático grande tenha cintura pequena ou vice-versa. Porém tal intuição está incorreta, e temos que existem grafos com número cromático e cintura arbitrariamente grandes.

Um primeiro passo para mostrar que existem grafos com número cromático e cintura arbitrariamente grandes é mostrar que existem grafos com número cromático arbitrariamente grande e livres de triângulos, ou seja, com cintura pelo menos $4$. Exemplos de construções de grafos livres de triângulos e com número cromático arbitrariamente grande incluem as construções de W. Tutte \cite{descartes1947three} e A. Zykov \cite{zykov1949some}, a construção de Mycielski \cite{mycielski1955coloriage}, e uma construção baseada em planos projetivos \cite{codenotti2000some}.

%Erd\H{o}s mostrou que para todo par de inteiros positivos $k$ e $g$, existe um grafo $G$ com número cromático pelo menos $k$ e cintura pelo menos $g$ \cite{erdos1959graph}. Dessa forma, podemos estudar propriedades que garantam a existência de grafos com cintura e número cromático grandes como subgrafos de outros grafos.

Erd\H{o}s mostrou que existem grafos com número cromático e cintura arbitrariamente grande \cite{erdos1959graph}.

\begin{teorema}\label{teoerdos}
Para todo par de inteiros $k,g > 0$, existe um grafo com número cromático pelo menos $k$ e cintura pelo menos $g$.
\end{teorema}

Uma forma de interpretar o teorema de Erd\H{o}s é, dado um inteiro $n_0 = n_0(k,g)$, todo $K_n$ com $n \geq n_0$ contém um subgrafo com número cromático pelo menos $k$ e cintura pelo menos $g$. Temos então que se um grafo contém um clique de tamanho suficientemente grande, o grafo contém um subgrafo com número cromático e cintura grandes.

Porém, garantir a existência de um clique como subgrafo não é simples, e diversos grafos de interesse não contém cliques grandes. Portanto, é interessante procurar outra condição que garanta a existência de subgrafos com número cromático e cintura grandes.

Em \cite{erdos1969conj}, Erd\H{o}s e Hajnal propõem a seguinte conjectura.

\begin{conjectura}[Erd\H{o}s e Hajnal, 1969]
Para todo par de inteiros positivos $k$ e $g$, existe um inteiro $f(k,g)$ tal que todo grafo $G$ com $\chi(G) \geq f(k,g)$ contém um subgrafo $H$ com $\chi(H) \geq k$ e $g(H) \geq g$.
\end{conjectura}

Ou seja, para garantir a existência de um subgrafo com número cromático e cintura grandes, basta que o número cromático do grafo seja suficientemente grande.

Podemos dizer que a conjectura é uma generalização do teorema de Erd\H{o}s, no sentido de que não precisamos de um clique grande para garantir a existência de um subgrafo com número cromático e cintura grandes, bastando apenas que o grafo ``se comporte como um clique'', ou seja, que tenha número cromático suficientemente grande para garantir a existência de um subgrafo com número cromático e cintura grandes.

Neste trabalho, estudaremos alguns resultados sobre famílias de grafos para as quais a conjectura de Erd\H{o}s e Hajnal é verdadeira. Em particular, no capítulo \ref{cap:livretriangulos}, mostraremos que para qualquer $k$ existe $f(k,4)$ \cite{rodl1977chromatic}, ou seja, que a conjectura de Erd\H{o}s e Hajnal é verdadeira se $g=4$. Assim como mostraremos que para qualquer $g$ existe $f(3,g)$.

No capítulo \ref{cap:kneser} mostraremos que os grafos de Kneser respeitam a conjectura de Erd\H{o}s e Hajnal \cite{mohar2015triangle}, especificamente mostraremos que grafos de Kneser suficientemente grandes contém \textit{blow-ups} de grafos de Kneser menores como subgrafos, e usando o Lema Local de Lovász mostraremos que \textit{blow-ups} de grafos de Kneser contém subgrafos com número cromático e cintura grandes.

No capítulo \ref{cap:shift} mostraremos que os \textit{shift graphs} respeitam a conjectura de Erd\H{o}s e Hajnal \cite{gabor2018cepa} por meio do Lema Local de Lovász usando um método semelhante, porém não idêntico, ao usado com os grafos de Kneser. Assim como apresentaremos alguns breves resultados sobre os \textit{type graphs}. % Detalhar